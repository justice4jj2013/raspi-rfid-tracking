\chapter{Implementation}
\label{ch:implementation}

This chapter details the hardware and software design of the system. Section \ref{sec:harddes} describes how the hardware devices are connected to build the system. Section \ref{sec:softdes} presents the software specifications, overall architecture, and individual components of the system.

This project explores the possibility of developing an RFID location sensing system using cost-effective hardware. This chapter details the software engineering tools and mathematical techniques that were employed to achieve this goal.



\section{Antenna Design}
\label{sec:antdes}

\section{Project management}

A number of considerations were taken into account when deciding how to manage this project. First, the system operates using a server-client model. This means that different software components are executing on multiple processing nodes. As a result, changes in one node need to be propagated in the whole system ensuring the consistency of the software.  Second, the software implementation is making use of different programming languages, multiple programming libraries, and a database management system. In order to ensure an iterative development process, where software components are constructed, debugged, and packaged together, it was decided to use the \textsc{Git} version control system\footnote{\textsc{Git} version control system - \url{http://git-scm.com/}}. This system keeps a distributed repository of all software and database files so that each node stores a copy of not only the whole software system, but also a complete history of changes. In addition, the use of a version control system stimulates the developer to merge a number of important changes into versions of the software. In this way, it becomes easy to track and monitor the project progress. The source code and documentation were hosted in a private repository on \textsc{GitHub}\footnote{The private project repository - \url{https://github.com/sandio/raspi-rfid-tracking}} with access granted to the people involved in developing and supervising the project.


\section{Software Engineering Practices}

A number of software engineering practices were of significant help when developing the RFID location sensing system. This section presents them and explains the problems that they solve.  

\subsection{Project decomposition}

It would have been a serious challenge to approach the project's task directly. The system consists of pieces of hardware that had to be orchestrated to solve a common problem. Therefore, it was very important to identify the system's components from early on. Hierarchical relationships between these parts were also defined. These steps ensured that the project could be divided into stages in order to systematically solve the main task. Regular deliveries of working components provided a more manageable way of constructing the final solution. For example, the work plan, devised before the start of the project, consisted of the following key activities:

\begin{enumerate}
 	\item Prepare the single-board computers
 	\item Construct functional RFID reader nodes
 	\item Receive information from the active RFID tag
 	\item Establish a network communication between nodes
 	\item Develop the localisation algorithm
 \end{enumerate}

Iterative construction of the system aided the development process. Problems and challenges were appearing gradually which helped solving them one at a time. 

\subsection{Object-oriented design}

This location sensing system is a combination of different software technologies. For instance, the system required an interface between a single-board computer and an RFID receiver. It also required means of communication between processing nodes. Logically, these and other requirements could be grouped into sets of functions, which is a motivation for employing an object-oriented design. This software methodology was used from the beginning of the project. Similar functionality is organised in a class. A class is responsible for all procedures concerning a particular part of the system. As a result, software is split into categories of functions, which makes it easy to address the class in charge of certain functionality.

Another benefit of the object-oriented design is modularity. For example, once input data is collected from all nodes it could be processed by a localisation algorithm in order to estimate the tag's position. Trilateration was chosen as the technique for computing locations. Object-oriented software development provides an easy way to experiment with different algorithms by substituting one class with another.

\subsection{System scalability}

In this project, three single-board computers collaborate by exchanging RFID readings to localise a tagged object. Three reference points are needed in order to use trilateration in two dimensions  \cite{Zhang2009}. Nevertheless, more reader nodes could be used, in case multileration is implemented, to give a better approximation of a tag's position. Another scenario involves nodes disconnecting and later reappearing into the network. These possible cases show the dynamic nature of the system. It could scale up as the system grows, but also scale down if a reader node is faulty. This is an important property of the system, which was noted at the start of the project. To ensure scalability of the server-client model, the multi-threading programming model was used. It allows multiple threads to exist within the context of a single process. As a result, the system could concurrently receive RFID measurements from multiple reader nodes, update data structures, and compute the location of the unknown object.

\subsection{Documentation}

Writing documentation was an important part of this project. The source code of the system has been systematically documented throughout the development process. Using the inline comments specifying how the software components work, an Application Programming Interface (API) was constructed using \textsc{Sphinx}\footnote{\textsc{Sphinx} - a Python documentation generator - \url{http://sphinx-doc.org/index.html}}, a \textsc{Python} documentation generator. The API contains specifications of data structures, variables, and functions. It is a valuable source of information that provides a quick reference of how the system's components work and interact with each other. In addition, a manual for future users of the system was written. It gives a quick introduction of how to set up and use the system. The API and user manual can be viewed in Appendix \textbf{REF}  \textbf{TODO}.

This project consists of both hardware and software components. In order to clearly understand how hardware components are connected and how software objects interact, a number of diagrams were used in Chapter \textbf{REF} and in the user manual. These diagrams were generated using \textsc{Blockdiag} \footnote{\textsc{Blockdiag} - simple diagram images generator - \url{http://blockdiag.com/en/}} , a diagram image generator written in \textsc{Python}.

\section{Summary}
