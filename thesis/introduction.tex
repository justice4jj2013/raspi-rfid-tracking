\chapter{Introduction}
\label{ch:introduction}

Radio frequency identification (RFID) is a technology that enables the recognition of objects tagged with an RFID transmitter. It functions by remotely obtaining data stored on RFID tags. This information is mainly used for identification purposes. Systems relying on such data can only provide course-grained location information \cite{Bouet2008}. Their RFID readers are positioned at strategic control points that sense tags entering their read range. However, if an object's identity is combined with its location, then the benefits of RFID could be greater. For example, patient care and hospital operations could be improved using remote identification and tracking of patients \cite{Cangialosi2007}. Also, the attendance of children at schools could be monitored \cite{Swartz2004}.

RFID localisation principles are similar to the ones used for indoor wireless networks \cite{Bouet2008}. There are certain differences between both technologies, which results in tracking methods that are altered to reflect the characteristics of RFID. This project uses the trilateration localisation technique to determine the position of a tag using three reader nodes in a controlled indoor environment.

RFID systems consist of tags and readers. While tags are simple devices, readers are more complex and usually require a connection to a host computer or network \cite{Landt2005}. The high costs of tags and readers are a major factor that constrains the penetration of this technology \cite{Want2006}. Nowadays, these devices are becoming affordable to users. In addition, the recent emergence of cheap and compact single-board computers, such as the Raspberry Pi\footnote{About the Raspberry Pi - \url{http://www.raspberrypi.org/about}}, creates an exiting opportunity to build a cost-effective RFID sensor network capable of localising tags. This can be realised by connecting readers to single-board computers to obtain the identity of tags and a measure of the signal strength, which could provide an estimation of the distance between receivers and transmitters.

On the one hand, the RFID technology has unprecedented advantages and it has gained the attention of big industries that have identified its potential \cite[p. 39]{Hunt2007}. On the other hand, the high costs of RFID tracking systems and components prevent most people from using and developing the technology\footnote{How much do RFID readers cost today? - \url{http://www.rfidjournal.com/faq/show?86}}. The hardware combination of affordable readers, tags, and single-board computers has the potential to benefit a vast range of businesses, but also do-it-yourself hobbyists and enthusiasts. This might result into improved automated handling and tracking of goods in a warehouse, for instance. It can also result in a fast-paced open-source development of the RFID technology applied in a wide range of scenarios. Constructing such a system is exiting because it can show that there can be cost-effective alternatives to commercial solutions, thus making the technology accessible to a wider audience.

\section{Hypothesis}

The hypothesis of this project is that existing localisation techniques can be applied to a cost-effective Raspberry-Pi sensor network to build a location sensing system achieving a localisation accuracy comparable to existing approaches. More specifically, the purpose of the project is to construct and programme three reader nodes, each consisting of an RFID reader connected to a single-board computer, cooperating in an indoor environment to estimate the position of a stationary or moving active tag based on the received signal strength indicator (RSSI) provided by the readers. This metric is converted to distance used by the trilateration geometrical process to compute the position of an unknown object.

\section{Results}

The above system was developed during the course of this project. A number of experiments measured the system's performance in terms of localisation error between the real and estimated positions of the tag. The average location error from these tests was 0.916 meters on each side of the object. These results are similar to the achievements of the previous work in this research field. Although the system could be tested in more scenarios under different conditions, these results confirmed that an affordable RFID positioning system can be built and operate with a reasonable performance.

\section{Thesis Outline}

This document consists of six chapters. The organisation is as follows:
\begin{itemize}
\item \textbf{Chapter \ref{ch:background}} gives background information on radio frequency identification, the hardware and  localisation algorithm used in this project. In addition, this chapter provides a literature review of previous work in this research field.
\item \textbf{Chapter \ref{ch:methodology}} describes an overview of the hardware setup of the system. In addition, it contains a definition of the problem this project is trying to solve. The rest of the chapter presents the overall design of the system and the methods for converting received signal strength indicator to distance. The trilateration geometrical process used to estimate the position of the tag is also described.
\item \textbf{Chapter \ref{ch:implementation}} gives implementation details of all parts of the system. This chapter lists the software engineering practices that helped develop the system. Any implementation difficulties are also discussed.
\item \textbf{Chapter \ref{ch:evaluation}} contains experiments that test the RFID hardware. In addition, the performance of the system is investigated and measured. The chapter includes a discussion, where results are compared with achievements of previous projects.
\item \textbf{Chapter \ref{ch:conclusion}} concludes this work by presenting the contributions of this project and suggesting future directions in which the system can be extended.
\end{itemize}

\section{Summary}

This chapter provided the motivation for this project. It also defined the hypothesis that is checked in this work. The main results were also given. Finally, the structure of this document was described. The next chapter provides background information on the technology and hardware used in this project. In addition, the previous work in this field is summarised.
